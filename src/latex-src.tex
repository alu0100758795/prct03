\documentclass[a4paper,12pt]{article}
\usepackage[spanish]{babel}
\usepackage[utf8]{inputenc}

\begin{document}

  Si simplemente se desea escribir texto normal en LaTeX,
  sin complicadas f\'ormulas matem\'aticas o efectos especiales
  como cambios de fuente, entonces simplemente tiene que escribir
  en espa\~nol normalmente.
  
  Si desea cambiar de p\'arrafo ha de dejar una l\'inea en blanco o bien
  utilizar el comando \par.
  No es necesario preocuparse de la sangr\'ia de los p\'arrafos:
  todos los p\'arrafos se sangrar\'an autom\'aticamente con la excepci\'on
  del primer p\'arrafo de una secci\'on.
  
  Se ha de distinguir entre la comilla simple ‘izquierda’
  y la comilla simple ‘derecha’ cuando se escribe en el ordenador.
  En el caso de que se quieran utilizar comillas dobles se han de
  escribir dos caracteres ‘comilla simple’ seguidos, esto es,
  ‘‘comillas dobles’’.
  
  Tambi\'en se ha de tener cuidado con los guiones: se utiliza un \'unico
  gui\'on para la separaci\'on de s\'ilabas, mientras que se utilizan
  tres guiones seguidos para producir un gui\'on de los que se usan
  como signo de puntuaci\'on --- como en esta oraci\'on.
  
  \title{Titulo del art\'iculo}
  \author{Nombre y Apellido \\ 
	  T\'ecnicas experimentales~\footnote{Universidad de la Laguna}
	  }
  \date{\today}
  \maketitle
  
  \begin{abstract}
   En \LaTeX{}~\cite{Lam:86} es sencillo escribir expresiones
   matem\'aticas como $a=\sum_{i=i}^{10} {x_i}^{3}$
   y deben ser escritas entre dos simbolos \$.
   Los super\'indices se obtienen con el simbolo \^{}, y
   los sub\'indices con el simbolo \_. 
   por ejemeplo: $x^2 \times y^{\alpha + \beta}$.
   tambi\'en se pueden escribir f\'ormulas centradas:
   \[h^2=a^2 + b^2\] 
  \end{abstract}
  
  \section{Primera seccion}
  
  \begin{thebibliography}{00}
   \bibitem{Lam:86}
    Lamport, Leslie
    TLA in picture
    \emph{IEE Transactions on Software Engineering},
    21(9), 768-775.
    (1995)
  \end{thebibliography}

\end{document}